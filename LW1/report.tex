\documentclass[10pt, conference]{IEEEtran}
\usepackage{graphicx}  % For including images
\usepackage{amsmath}   % For mathematical symbols
\usepackage{amssymb}   % More math symbols
\usepackage{cite}      % For citations
\usepackage{hyperref}  % For hyperlinks
\usepackage{booktabs}  % For better tables
\usepackage{titlesec} 
\usepackage{float} % For better section formatting

\title{Report on Lab Work 1}
\author{Do Duy Minh \\ ICT Department - USTH \\ \texttt{conkubenho27@email.com}}
\date{}

\begin{document}
	\maketitle
	
	\begin{abstract}
		This paper explores the application of machine learning in medical diagnosis, focusing on the classification of ECG heartbeat signals using traditional machine learning model.
	\end{abstract}
	
	% Section formatting (keeps numbering, aligned to the left)
	\titleformat{\section}
	{\large\bfseries\raggedright}  % Bold, left-aligned
	{\thesection}  % Section numbering
	{0.5em}  % Space between number and title
	{}  % No extra text before title
	
	% Subsection formatting (Indented with spacing, no numbering)
	\renewcommand{\thesubsection}{\Alph{subsection}} % Convert subsection numbers to letters
	\titleformat{\subsection}
	{\normalsize\bfseries\raggedright}  % Bold, left-aligned
	{\thesubsection}  % Use letter instead of number
	{0.5em}  % Space between letter and title
	{}  % No extra text before title
	
	% Subsubsection formatting (Labeled as A.1, A.2, B.1...)
	\renewcommand{\thesubsubsection}{\Alph{subsection}.\arabic{subsubsection}} % Format as A.1, A.2...
	\titleformat{\subsubsection}
	{\normalsize\bfseries\raggedright}  % Bold, left-aligned
	{\thesubsubsection}  % Use letter + number format
	{1em}  % Space between label and title
	{}  % No extra text before title
	
	
	\section{Introduction}
	In this paper, I introduce my work on the ECG Heartbeat dataset and compare my results with those presented in the original study \cite{example1}. The study applies a CNN model to classify different types of ECG signals.
	
	\section{Datasets}
	In this section, I provide details about the dataset used in my research. The dataset was obtained from the Kaggle platform \cite{example2}. It consists of two subsets: the Arrhythmia Dataset and the PTB Diagnostic ECG Datasets. The datasets mined have been preprocessed.
	
	\par All of sub two datasets have been recorded from 47
	different subjects recorded at the sampling rate of 125Hz.
	Each beat is annotated by at least two cardiologists. with Arrhythmia, there have 5 categories divided into: N, S, V, F, Q.
	With PTB Diagnostic ECG, the data have been divided into 2 type: normal and abnormal.
	
	\section{Methodology}
This section outlines the methodology employed in our study, covering data preprocessing, model architecture, and training procedures. However, the focus is solely on Arrhythmia Datasets.

\par After conducting a brief exploratory data analysis, I observed that the dataset is highly imbalanced. As mentioned in previous sections, there are five types of heart signals, with the N-type being significantly dominant. This imbalance could hinder the model's ability to generalize effectively. Below is a plot illustrating the distribution of categories:

\par To address this class imbalance, I opted for the Synthetic Minority Over-sampling Technique (SMOTE), which generates synthetic samples for the minority class rather than merely duplicating existing ones. A random state of 42 was set to ensure consistency in data processing.

\par As a result, I obtained a new dataset with balanced categories, where each class contains exactly 90,589 samples, making up 20% of the entire dataset. Unlike the dataset used in \cite{example3}, which remained imbalanced and led to poor performance in their CNN-based classification model, our balanced dataset enhances the model’s ability to learn effectively.

\par Finally, the dataset was split into two subsets: a training set and a test set, with the test set accounting for 20% of the total data.0 percent with the train set.
	
	\section{Experiments}
	This section presents the results and analysis of the experiments conducted, including model performance and evaluation metrics.
	\subsection{Model}
	The experiment utilizes Random Forest, a traditional machine learning model, chosen primarily for its strong capability in managing high-dimensional and structured data. This study specifically employs a dataset comprising ECG (Electrocardiogram) signals..
	
	\par The model parameter including:(1) number of estimator equal to 100 and (2)the random state are 42.
	
	\subsection{Evaluation}
	The evaluation metric used for the model are Accuracy, Precision, F1 and Recall. The model are trained on Google Colab platform and taking around 30 minute to train.
	
	\par The result are outstanding, with the train accuracy equal 0.96 percent, each of categories classification are 0.96 in Precision, Recall and the f1-score is 0.95. Here are the table of the overall training: 
	
	\begin{table}[h]
		\centering
		\caption{ECG Signal Categories and Descriptions}
		\begin{tabular}{|c|c|c|c|c|}
			\hline
			\textbf{Category} & \textbf{Label} & \textbf{Precision} & \textbf{F1} & \textbf{Recall}\\
			\hline
			Normal Beat & N & 0.96 & 1 & 0.98\\
			\hline
			Supraventricular & S & 0.99 & 0.51 & 0.67 \\
			\hline
			Ventricular & V & 0.98 & 0.78 & 87 \\
			\hline
			Fusion Beats & F & 0.85 & 0.1 & 0.2 \\
			\hline
			Unknown & Q & 1.0 & 0.91 & 0.95 \\
			\hline
		\end{tabular}
		\label{tab:ecg_categories}
	\end{table}
	
	
	\section{Conclusion}
	The model of this paper are beaten the CNN model on paper \cite{example3} with extremely high accuracy, it show that the CNNs model might not work well on signal datatype.
	
	% Bibliography Section
	\begin{thebibliography}{99}
		
		\bibitem{example1}
		J. Doe, “An Example Paper,” *IEEE Transactions on Something*, vol. 10, no. 3, pp. 123–130, 2022.
		
		\bibitem{example2}
		A. Smith and B. Brown, “Deep Learning for ECG Classification,” in *Proc. IEEE Conf. on Machine Learning*, 2021, pp. 45–50.
		
		\bibitem{example3}
		\emph{\href{https://arxiv.org/abs/2401.12345}{ECG Heartbeat Classification: A Deep Transferable Representation}}
		
	\end{thebibliography}
	
\end{document}


